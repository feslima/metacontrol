% Encoding: UTF-8
@article{alves2018,
author = {Alves, Victor
M. C. and Lima, Felipe S. and Silva, Sidinei K. and Araujo, Antonio C. B.},
title = {Metamodel-Based Numerical Techniques for Self-Optimizing Control},
journal = {Industrial \& Engineering Chemistry Research},
volume = {57},
number = {49},
pages = {16817-16840},
year = {2018},
doi = {10.1021/acs.iecr.8b04337},
URL = {https://doi.org/10.1021/acs.iecr.8b04337},
eprint = {https://doi.org/10.1021/acs.iecr.8b04337}
}

@Article{Liu2019,
author  = {Liu, Kaile and Jin, Bo and Zhao, Yunlei and Liang, Zhiwu},
title   = {Self-Optimizing Control Structure and Dynamic Behavior for CO2 Compression and Purification Unit in Oxy-fuel Combustion Application},
journal = {Industrial \& Engineering Chemistry Research},
year    = {2019},
volume  = {58},
number  = {8},
pages   = {3199-3210},
doi     = {10.1021/acs.iecr.9b00121},
eprint  = {https://doi.org/10.1021/acs.iecr.9b00121},
url     = {https://doi.org/10.1021/acs.iecr.9b00121},
}
@article{dillon2005,
title={Oxy-combustion processes for CO2 capture from power plant},
author={Dillon, DJ and White, V and Allam, RJ and Wall, RA and Gibbins, JIEA},
journal={Engineering investigation report},
volume={9},
year={2005}
}

@article{JIN2015,
title = "Optimization and control for CO2 compression and purification unit in oxy-combustion power plants",
journal = "Energy",
volume = "83",
pages = "416 - 430",
year = "2015",
issn = "0360-5442",
doi = "https://doi.org/10.1016/j.energy.2015.02.039",
url = "http://www.sciencedirect.com/science/article/pii/S0360544215001942",
author = "Bo Jin and Haibo Zhao and Chuguang Zheng",
keywords = "CO compression and purification unit, Dynamic simulation, Control system design, Oxy-fuel combustion",
abstract = "High CO2 purity products can be obtained from oxy-combustion power plants through a CO2 CPU (compression and purification unit) based on phase separation method. To ensure that CPU (with double flash separators) can be operated under optimal conditions, this paper focuses on single variable analysis, multi-variable optimization, dynamic simulation and control system design for CPU in oxy-combustion power plants. It is found that optimal operating conditions are 30 bar, 30.42 °C, −24.64 °C, and −55 °C for multi-stage CO2 compressor discharge pressure, flue gas temperature after compression, first flash separator temperature, and second flash separator temperature, respectively. The designed double temperature control structure based on a systematic top-down analysis and bottom-up design method is more suitable than the single temperature control structure under different operating scenarios (load change and flue gas composition ramp change). To bear operating disturbances, operating strategies like elevating temperature, manipulating valves and adjusting setpoints are proposed. Influence of SOx would be more obvious than that of NOx, whilst the kij mixing parameters in Peng–Robinson property method affects little on process optimization and control system design. Comprehensive dynamic model with specified control system provides possibility to integrate CPU with full-train oxy-combustion power plants."
}

@article{TOFTEGAARD2010581,
	title = "Oxy-fuel combustion of solid fuels",
	journal = "Progress in Energy and Combustion Science",
	volume = "36",
	number = "5",
	pages = "581 - 625",
	year = "2010",
	issn = "0360-1285",
	doi = "https://doi.org/10.1016/j.pecs.2010.02.001",
	url = "http://www.sciencedirect.com/science/article/pii/S0360128510000201",
	author = "Maja B. Toftegaard and Jacob Brix and Peter A. Jensen and Peter Glarborg and Anker D. Jensen",
	keywords = "Carbon capture and storage, Oxy-fuel combustion, Coal, Biomass, Emissions",
	abstract = "Oxy-fuel combustion is suggested as one of the possible, promising technologies for capturing CO2 from power plants. The concept of oxy-fuel combustion is removal of nitrogen from the oxidizer to carry out the combustion process in oxygen and, in most concepts, recycled flue gas to lower the flame temperature. The flue gas produced thus consists primarily of carbon dioxide and water. Much research on the different aspects of an oxy-fuel power plant has been performed during the last decade. Focus has mainly been on retrofits of existing pulverized-coal-fired power plant units. Green-field plants which provide additional options for improvement of process economics are however likewise investigated. Of particular interest is the change of the combustion process induced by the exchange of carbon dioxide and water vapor for nitrogen as diluent. This paper reviews the published knowledge on the oxy-fuel process and focuses particularly on the combustion fundamentals, i.e. flame temperatures and heat transfer, ignition and burnout, emissions, and fly ash characteristics. Knowledge is currently available regarding both an entire oxy-fuel power plant and the combustion fundamentals. However, several questions remain unanswered and more research and pilot plant testing of heat transfer profiles, emission levels, the optimum oxygen excess and inlet oxygen concentration levels, high and low-temperature fire-side corrosion, ash quality, plant operability, and models to predict NOx and SO3 formation is required."
}
@article{BUHRE2005283,
	title = "Oxy-fuel combustion technology for coal-fired power generation",
	journal = "Progress in Energy and Combustion Science",
	volume = "31",
	number = "4",
	pages = "283 - 307",
	year = "2005",
	issn = "0360-1285",
	doi = "https://doi.org/10.1016/j.pecs.2005.07.001",
	url = "http://www.sciencedirect.com/science/article/pii/S0360128505000225",
	author = "B.J.P. Buhre and L.K. Elliott and C.D. Sheng and R.P. Gupta and T.F. Wall",
	keywords = "Oxy-fuel combustion, pf coal combustion, CO capture",
	abstract = "The awareness of the increase in greenhouse gas emissions has resulted in the development of new technologies with lower emissions and technologies that can accommodate capture and sequestration of carbon dioxide. For existing coal-fired combustion plants there are two main options for CO2 capture: removal of nitrogen from flue gases or removal of nitrogen from air before combustion to obtain a gas stream ready for geo-sequestration. In oxy-fuel combustion, fuel is combusted in pure oxygen rather than air. This technology recycles flue gas back into the furnace to control temperature and makeup the volume of the missing N2 to ensure there is sufficient gas to maintain the temperature and heat flux profiles in the boiler. A further advantage of the technology revealed in pilot-scale tests is substantially reduced NOx emissions. For coal-fired combustion, the technology was suggested in the eighties, however, recent developments have led to a renewed interest in the technology. This paper provides a comprehensive review of research that has been undertaken, gives the status of the technology development and assessments providing comparisons with other power generation options, and suggests research needs."
}

@article{POSCH2012254,
title = "Optimization of CO2 compression and purification units (CO2CPU) for CCS power plants",
journal = "Fuel",
volume = "101",
pages = "254 - 263",
year = "2012",
note = "8th European Conference on Coal Research and Its Applications",
issn = "0016-2361",
doi = "https://doi.org/10.1016/j.fuel.2011.07.039",
url = "http://www.sciencedirect.com/science/article/pii/S0016236111004364",
author = "Sebastian Posch and Markus Haider",
keywords = "CCS, carbon capture and storage, CO compression, CO quality, Oxy-fuel, Purification",
abstract = "Oxy-fuel power plants are among the currently known major carbon capture and storage (CCS) technologies those which produce the lowest carbon dioxide (CO2) purity CO2-stream. As a consequence, the oxy-fuel technology has the highest requirement towards CO2 purification. In the current paper two different purification processes are modeled in Aspen Plus™ based on a Peng–Robinson property method with kij mixing coefficients. Both separation processes are based on phase separation technique. The first one is equipped with two flash columns and the second one includes a distillation column. Both plants require a dehydration plant for water removal. For each process the impact of main design parameters on performance features such as specific power requirement, specific cooling duty, separation efficiency and CO2 purity are analyzed, including variations in flue gas composition, plant load and kij mixing parameters. The results were compared to a conventional 5-stage CO2 compressor."
}

@article{KOOHESTANIAN2017570,
	title = "Sensitivity analysis and multi-objective optimization of CO2CPU process using response surface methodology",
	journal = "Energy",
	volume = "122",
	pages = "570 - 578",
	year = "2017",
	issn = "0360-5442",
	doi = "https://doi.org/10.1016/j.energy.2017.01.129",
	url = "http://www.sciencedirect.com/science/article/pii/S0360544217301366",
	author = "Esmaeil Koohestanian and Abdolreza Samimi and Davod Mohebbi-Kalhori and Jafar Sadeghi",
	keywords = "CO capture, Multi-objective optimization, Sensitivity analysis, RSM methodology, Oxy-fuel combustion",
	abstract = "Compression and purification unit (CPU) is a common industrial process for capturing CO2 from oxy-fuel combustion where high energy requirement is one of its disadvantages. This study focuses on analyzing of the sensitivity and optimizing multi-objectively the operating conditions of CPU, using response surface methodology (RSM). The main objective was to increase the efficiency of CO2 removal from the oxy-fuel combustion power plant. Statistical analysis reveals that reducing the first separator temperature, not only, plays a major role in CO2 separation, but also, it decreases the total work and heat duty of the process. It was found that the optimal multi-stage CO2 compressor discharge pressure was 25.34 bar while regular pressure for this process was reported as 30 bar. Furthermore, the optimal flue gas temperature before, between and after compression, and the first and second flash separator temperatures were 20 °C, 20 °C, 20.6 °C, −38.2 °C, and −55 °C, respectively. with the previous works carried out in a constant amount of CO2 separation, the proposed process leads to lower pressure, and therefore lower operating and capital costs."
}

@article{caballero2008,
	author = {Caballero, José A. and Grossmann, Ignacio E.},
	title = {An algorithm for the use of surrogate models in modular flowsheet optimization},
	journal = {AIChE Journal},
	volume = {54},
	number = {10},
	pages = {2633-2650},
	keywords = {simulation, process, design (process simulation), mathematical modeling, numerical solutions, optimization},
	doi = {10.1002/aic.11579},
	url = {https://aiche.onlinelibrary.wiley.com/doi/abs/10.1002/aic.11579},
	eprint = {https://aiche.onlinelibrary.wiley.com/doi/pdf/10.1002/aic.11579},
	abstract = {Abstract In this work a methodology is presented for the rigorous optimization of nonlinear programming problems in which the objective function and (or) some constraints are represented by noisy implicit black box functions. The special application considered is the optimization of modular process simulators in which the derivatives are not available and some unit operations introduce noise preventing the calculation of accurate derivatives. The black box modules are substituted by metamodels based on a kriging interpolation that assumes that the errors are not independent but a function of the independent variables. A Kriging metamodel uses non-Euclidean measure of distance to avoid sensitivity to the units of measure. It includes adjustable parameters that weigh the importance of each variable for obtaining a good model representation, and it allows calculating errors that can be used to establish stopping criteria and provide a solid base to deal with “possible infeasibility” due to inaccuracies in the metamodel representation of objective function and constraints. The algorithm continues with a refining stage and successive bound contraction in the domain of independent variables with or without kriging recalibration until an acceptable accuracy in the metamodel is obtained. The procedure is illustrated with several examples. © 2008 American Institute of Chemical Engineers AIChE J, 2008},
	year = {2008}
}

@Article{wachter2006,
author="W{\"a}chter, Andreas
and Biegler, Lorenz T.",
title="On the implementation of an interior-point filter line-search algorithm for large-scale nonlinear programming",
journal="Mathematical Programming",
year="2006",
month="Mar",
day="01",
volume="106",
number="1",
pages="25--57",
abstract="We present a primal-dual interior-point algorithm with a filter line-search method for nonlinear programming. Local and global convergence properties of this method were analyzed in previous work. Here we provide a comprehensive description of the algorithm, including the feasibility restoration phase for the filter method, second-order corrections, and inertia correction of the KKT matrix. Heuristics are also considered that allow faster performance. This method has been implemented in the IPOPT code, which we demonstrate in a detailed numerical study based on 954 problems from the CUTEr test set. An evaluation is made of several line-search options, and a comparison is provided with two state-of-the-art interior-point codes for nonlinear programming.",
issn="1436-4646",
doi="10.1007/s10107-004-0559-y",
url="https://doi.org/10.1007/s10107-004-0559-y"
}

@book{DACE,
title = "DACE - A Matlab Kriging Toolbox, Version 2.0",
author = "Lophaven, {S{\o}ren Nymand} and Nielsen, {Hans Bruun} and Jacob S{\o}ndergaard",
year = "2002",
publisher = "DTU Orbit",
language = "English"
}

@article{hori05,
author = {Hori, Eduardo S. and Skogestad, Sigurd and Alstad, Vidar},
title = {Perfect Steady-State Indirect Control},
journal = {Industrial \& Engineering Chemistry Research},
volume = {44},
number = {4},
pages = {863-867},
year = {2005},
doi = {10.1021/ie049736l},
URL = {https://doi.org/10.1021/ie049736l},
eprint = {https://doi.org/10.1021/ie049736l}
}

@article{alstad09,
title = "Optimal measurement combinations as controlled variables",
journal = "Journal of Process Control",
volume = "19",
number = "1",
pages = "138 - 148",
year = "2009",
issn = "0959-1524",
doi = "https://doi.org/10.1016/j.jprocont.2008.01.002",
url = "http://www.sciencedirect.com/science/article/pii/S0959152408000073",
author = "Vidar Alstad and Sigurd Skogestad and Eduardo S. Hori",
keywords = "Control structure selection, Self-optimizing control, Measurement selection, Process control, Quadratic optimization, Plantwide control",
abstract = "This paper deals with the optimal selection of linear measurement combinations as controlled variables, c=Hy. The objective is to achieve “self-optimizing control”, which is when fixing the controlled variables c indirectly gives near-optimal steady-state operation with a small loss. The nullspace method of Alstad and Skogestad [V. Alstad, S. Skogestad, Null space method for selecting optimal measurement combinations as controlled variables, Ind. Eng. Chem. Res. 46 (3) (2007) 846–853] focuses on minimizing the loss caused by disturbances. We here provide an explicit expression for H for the case where the objective is to minimize the combined loss for disturbances and measurement errors. In addition, we extend the nullspace method to cases with extra measurements by using the extra degrees of freedom to minimize the loss caused by measurement errors. Finally, the results are interpreted more generally as deriving linear invariants for quadratic optimization problems."
}

@article{halvorsen03,
author = {Halvorsen, Ivar J. and Skogestad, Sigurd and Morud, John C. and Alstad, Vidar},
title = {Optimal Selection of Controlled Variables},
journal = {Industrial \& Engineering Chemistry Research},
volume = {42},
number = {14},
pages = {3273-3284},
year = {2003},
doi = {10.1021/ie020833t},
URL = {https://doi.org/10.1021/ie020833t},
eprint = { 
        https://doi.org/10.1021/ie020833t}
}

@article{hori08,
author = {Hori, Eduardo Shigueo and Skogestad, Sigurd},
title = {Selection of Controlled Variables: Maximum Gain Rule and Combination of Measurements},
journal = {Industrial \& Engineering Chemistry Research},
volume = {47},
number = {23},
pages = {9465-9471},
year = {2008},
doi = {10.1021/ie0711978},

URL = {https://doi.org/10.1021/ie0711978},
eprint = {https://doi.org/10.1021/ie0711978}
}

@article{skoge00,
title = "Plantwide control: the search for the self-optimizing control structure",
journal = "Journal of Process Control",
volume = "10",
number = "5",
pages = "487 - 507",
year = "2000",
issn = "0959-1524",
doi = "https://doi.org/10.1016/S0959-1524(00)00023-8",
url = "http://www.sciencedirect.com/science/article/pii/S0959152400000238",
author = "Sigurd Skogestad",
abstract = "Plantwide control is concerned with the structural decisions involved in the control system design of a chemical plant (C.S. Foss, Critique of chemical process control theory, AIChE Journal 19(2), 1973) 209–214; “Which variables should be controlled, which variables should be measured, which inputs should be manipulated, and which links should be made between them?” In particular, the first issue about which variables to control has received little attention. It is argued that the answer is related to finding a simple and robust way of implementing the economically optimal operating policy. The goal is to find a set of controlled variables which, when kept at constant setpoints, indirectly lead to near-optimal operation with acceptable loss. This is denoted “self-optimizing” control. Since the economics are determined by the overall plant behavior, it is necessary to take a plantwide perspective. A systematic procedure for finding suitable controlled variables based on only steady-state information is presented. Important steps are degree of freedom analysis, definition of optimal operation (cost and constraints), and evaluation of the loss when the controlled variables are kept constant rather than optimally adjusted. A case study yields very interesting insights into the control and maximum throughput of distillation columns."
}

@article{luyben2006,
title = "Evaluation of criteria for selecting temperature control trays in distillation columns",
journal = "Journal of Process Control",
volume = "16",
number = "2",
pages = "115 - 134",
year = "2006",
issn = "0959-1524",
doi = "https://doi.org/10.1016/j.jprocont.2005.05.004",
url = "http://www.sciencedirect.com/science/article/pii/S0959152405000569",
author = "William L. Luyben",
abstract = "The use of tray temperatures to infer compositions is widespread in distillation control. A number of criteria have been proposed for selecting which trays to hold at constant temperature. The most commonly used are (1) choosing a tray where there are large changes in temperature from tray to tray (“slope” of the temperature profile), (2) finding the tray where there is the largest change in temperature for a change in the manipulated variable (“sensitivity”), (3) using singular value decomposition (SVD) analysis, (4) selecting the tray where the temperature does not change as feed composition changes while producing the desired distillate and bottoms purities and (5) choosing the tray that produces the smallest changes in product purities when it is held constant in the face of feed composition disturbances. This paper provides a quantitative comparison of the effectiveness of these five alternative criteria. Several systems are considered, ranging from ideal binary to azeotropic multi-component. Results show that SVD analysis provides a simple and effective method for selecting temperature control tray location."
}

@article{morari1980,
title={Studies in the synthesis of control structures for chemical processes: Part I: Formulation of the problem. Process decomposition and the classification of the control tasks. Analysis of the optimizing control structures},
author={Morari, Manfred and Arkun, Yaman and Stephanopoulos, George},
journal={AIChE Journal},
volume={26},
number={2},
pages={220--232},
year={1980},
publisher={Wiley Online Library}
}

@book{python2009,
 author = {Van Rossum, Guido and Drake, Fred L.},
 title = {Python 3 Reference Manual},
 year = {2009},
 isbn = {1441412697},
 publisher = {CreateSpace},
 address = {Scotts Valley, CA}
}

@article{kariwala2009,
  title={Bidirectional branch and bound for controlled variable selection. Part II: Exact local method for self-optimizing control},
  author={Kariwala, Vinay and Cao, Yi},
  journal={Computers and chemical engineering},
  volume={33},
  number={8},
  pages={1402--1412},
  year={2009},
  publisher={Elsevier}
}

@article{krige1951,
  title={A statistical approach to some basic mine valuation problems on the Witwatersrand},
  author={Krige, Daniel G},
  journal={Journal of the Southern African Institute of Mining and Metallurgy},
  volume={52},
  number={6},
  pages={119--139},
  year={1951},
  publisher={Southern African Institute of Mining and Metallurgy}
}

@article{sacks1989,
  title={Design and analysis of computer experiments},
  author={Sacks, Jerome and Welch, William J and Mitchell, Toby J and Wynn, Henry P},
  journal={Statistical science},
  pages={409--423},
  year={1989},
  publisher={JSTOR}
}

@inproceedings{alexandrov2000,
  title={Optimization with variable-fidelity models applied to wing design},
  author={Alexandrov, N and Lewis, R and Gumbert, C and Green, L and Newman, P},
  booktitle={38th aerospace sciences meeting and exhibit},
  pages={841},
  year={2000}
}

@article{jones2001,
  title={A taxonomy of global optimization methods based on response surfaces},
  author={Jones, Donald R},
  journal={Journal of global optimization},
  volume={21},
  number={4},
  pages={345--383},
  year={2001},
  publisher={Springer}
}

@book{forrester2008,
title={Engineering design via surrogate modelling: a practical guide},
author={Forrester, Alexander and Sobester, Andras and Keane, Andy},
year={2008},
publisher={John Wiley & Sons}
}

@PhdThesis{sasena2002,
author = {Michael James Sasena},
title  = {Flexibility and efficiency enhancements for constrained global design optimization with kriging approximations},
year   = {2002},
school = {University of Michigan Ann Arbor, MI},
}

@inbook{umar12,
  author={Umar, L M and Hu, W and Cao, Y and Kariwala, V},
  chapter = {Selection of Controlled Variables using Self-optimizing Control Method},
  title={Plantwide Control: Recent Developments and Applications},
  editor={Rangaiah, G P and Kariwala, V},
  year={2012},
  publisher={John Wiley \& Sons},
  adress = {Chichester, United Kingdom}
}

@article{araujo07,
title = "Application of plantwide control to the HDA process. I—steady-state optimization and self-optimizing control",
journal = "Control Engineering Practice",
volume = "15",
number = "10",
pages = "1222 - 1237",
year = "2007",
note = "Special Issue - International Symposium on Advanced Control of Chemical Processes (ADCHEM)",
issn = "0967-0661",
doi = "https://doi.org/10.1016/j.conengprac.2006.10.014",
url = "http://www.sciencedirect.com/science/article/pii/S0967066106001997",
author = "Antonio C.B. de Araújo and Marius Govatsmark and Sigurd Skogestad",
keywords = "HDA process, Self-optimizing control, Selection of controlled variable, Aspen Plus",
abstract = "This paper describes the application of self-optimizing control to a large-scale process, the HDA plant. The idea is to select controlled variables which when kept constant lead to minimum economic loss. First, the optimal active constraints need to be controlled. Next, controlled variables need to be found for the remaining unconstrained degrees of freedom. In order to avoid the combinatorial problem related to the selection of outputs/measurements for such large plants, a local (linear) analysis based on singular value decomposition (SVD) is used for pre-screening. This is followed by a more detailed analysis using the nonlinear model. Note that a steady-state model, in this case one built in Aspen PlusTM, is sufficient for selecting controlled variables. A dynamic model is required to design and test the complete control system which include regulatory control. This is considered in the part II of the series."
}

@book{skogebook,
   title =     {Multivariable Feedback Control: Analysis and Design},
   author =    {Sigurd Skogestad and Ian Postlethwaite},
   publisher = {Wiley-Interscience},
   year =      {1996},
   series =    {},
   edition =   {1},
   volume =    {},
}

@article{alstad07,
author = {Alstad, Vidar and Skogestad, Sigurd},
title = {Null Space Method for Selecting Optimal Measurement Combinations as Controlled Variables},
journal = {Industrial \& Engineering Chemistry Research},
volume = {46},
number = {3},
pages = {846-853},
year = {2007},
doi = {10.1021/ie060285+},

URL = {https://doi.org/10.1021/ie060285+},
eprint = { 
        https://doi.org/10.1021/ie060285+}

}

@article{hori07,
title = "Selection of Control Structure and Temperature Location for Two-Product Distillation Columns",
journal = "Chemical Engineering Research and Design",
volume = "85",
number = "3",
pages = "293 - 306",
year = "2007",
issn = "0263-8762",
doi = "https://doi.org/10.1205/cherd06115",
url = "http://www.sciencedirect.com/science/article/pii/S0263876207730514",
author = "E.S. Hori and S. Skogestad",
keywords = "distillation column, multicomponent distillation, control structure selection",
abstract = "The choice of control structures for distillation columns is important for practical industrial operation. There is no single ‘best’ structure for all columns, so some authors feel that each column should be treated independently. Nevertheless, the objective of this work is to find a structure that is ‘reasonable’ for all columns. In this paper, we consider the steady-state deviations in product compositions in response to disturbances, assuming that only flows and temperatures are available for control. For most columns a good choice is to fix reflux and a temperature. For binary separations, the temperature should be located where the temperature slope is steep. For multicomponent mixtures, the same rule applies except that one should avoid column sections with large changes in non-key components, for example at the column end and at the feed. Control of two temperatures is better for some columns, but not all."
}

@article{kariwala08,
author = {Kariwala, Vinay and Cao, Yi and Janardhanan, S.},
title = {Local Self-Optimizing Control with Average Loss Minimization},
journal = {Industrial \& Engineering Chemistry Research},
volume = {47},
number = {4},
pages = {1150-1158},
year = {2008},
doi = {10.1021/ie070897+},

URL = {https://doi.org/10.1021/ie070897+},
eprint = {https://doi.org/10.1021/ie070897+}
}

@misc{qtframework2017, 
title={{The Qt Framework}}, 
url={https://www.qt.io/qt-framework}, 
journal={The Qt Framework}, 
author= {{The Qt Company Ltd}}, 
year={2017}, 
month={Oct}
}

@misc{cyipopt,
author       = {Kummerer, Matthias and Moore, Jason K.},
title        = {{Cython interface for the interior point optimzer IPOPT}},
month        = {December},
year         = {2019},
version      = {0.2.0},
url          = {https://github.com/matthias-k/cyipopt}
}

@article{rasmussen2010,
  title={Gaussian processes for machine learning (GPML) toolbox},
  author={Rasmussen, Carl Edward and Nickisch, Hannes},
  journal={Journal of machine learning research},
  volume={11},
  number={Nov},
  pages={3011--3015},
  year={2010}
}

@article{skoge04,
title = "Control structure design for complete chemical plants",
journal = "Computers & Chemical Engineering",
volume = "28",
number = "1",
pages = "219 - 234",
year = "2004",
note = "Escape 12",
issn = "0098-1354",
doi = "https://doi.org/10.1016/j.compchemeng.2003.08.002",
url = "http://www.sciencedirect.com/science/article/pii/S0098135403001984",
author = "Sigurd Skogestad",
keywords = "Control structure design, Chemical plants, Plantwide control, Process control",
abstract = "Control structure design deals with the structural decisions of the control system, including what to control and how to pair the variables to form control loops. Although these are very important issues, these decisions are in most cases made in an ad hoc fashion, based on experience and engineering insight, without considering the details of each problem. In the paper, a systematic procedure for control structure design for complete chemical plants (plantwide control) is presented. It starts with carefully defining the operational and economic objectives, and the degrees of freedom available to fulfill them. Other issues, discussed in the paper, include inventory and production rate control, decentralized versus multivariable control, loss in performance by bottom-up design, and a definition of a the “complexity number” for the control system."
}

@article{cao05,
title = "Improved branch and bound method for control structure screening",
journal = "Chemical Engineering Science",
volume = "60",
number = "6",
pages = "1555 - 1564",
year = "2005",
issn = "0009-2509",
doi = "https://doi.org/10.1016/j.ces.2004.10.025",
url = "http://www.sciencedirect.com/science/article/pii/S0009250904008462",
author = "Yi Cao and Prabirkumar Saha",
keywords = "Control structure screening, Branch and bound method, Hankel singular value, Tennessee–Eastman process, Process control, Stability, Systems engineering, Optimization",
abstract = "The main aim of this paper is to present an improved algorithm of “Branch and Bound” method for control structure screening. The new algorithm uses a best-first search approach, which is more efficient than other algorithms based on depth-first search approaches. Detailed explanation of the algorithms is provided in this paper along with a case study on Tennessee–Eastman process to justify the theory of branch and bound method. The case study uses the Hankel singular value to screen control structure for stabilization. The branch and bound method provides a global ranking to all possible input and output combinations. Based on this ranking an efficient control structure with least complexity for stabilizing control is detected which leads to a decentralized proportional controller."
}


@article{cao08,
title = "Bidirectional branch and bound for controlled variable selection: Part I. Principles and minimum singular value criterion",
journal = "Computers & Chemical Engineering",
volume = "32",
number = "10",
pages = "2306 - 2319",
year = "2008",
issn = "0098-1354",
doi = "https://doi.org/10.1016/j.compchemeng.2007.11.011",
url = "http://www.sciencedirect.com/science/article/pii/S0098135407002906",
author = "Yi Cao and Vinay Kariwala",
keywords = "Branch and bound, Control structure design, Controlled variables, Combinatorial optimization, Minimum singular value, Self-optimizing control",
abstract = "The minimum singular value (MSV) rule is a useful tool for selecting controlled variables (CVs) from the available measurements. However, the application of the MSV rule to large-scale problems is difficult, as all feasible measurement subsets need to be evaluated to find the optimal solution. In this paper, a new and efficient branch and bound (BAB) method for selection of CVs using the MSV rule is proposed by posing the problem as a subset selection problem. In traditional BAB algorithms for subset selection problems, pruning is performed downwards (gradually decreasing subset size). In this work, the branch pruning is considered in both upward (gradually increasing subset size) and downward directions simultaneously so that the total number of subsets evaluated is reduced dramatically. Furthermore, a novel bidirectional branching strategy to dynamically branch solution trees for subset selection problems is also proposed, which maximizes the number of nodes associated with the branches to be pruned. Finally, by replacing time-consuming MSV calculations with novel determinant based conditions, the efficiency of the bidirectional BAB algorithm is increased further. Numerical examples show that with these new approaches, the CV selection problem can be solved incredibly fast."
}

@techreport{dacereport,
author       = "S. N. Lophaven and H. B. Nielsen and J. S{\o}ndergaard",
title        = "Aspects of the Matlab toolbox {DACE}",
year         = "2002",
number       = "",
series       = "IMM-TR-2002-13",
institution  = "Informatics and Mathematical Modelling, Technical University of Denmark, {DTU}",
address      = "Richard Petersens Plads, Building 321, {DK-}2800 Kgs. Lyngby",
type         = ""
}

@article{gera13,
author = {Gera, Vivek and Panahi, Mehdi and Skogestad, Sigurd and Kaistha, Nitin},
title = {Economic Plantwide Control of the Cumene Process},
journal = {Industrial \& Engineering Chemistry Research},
volume = {52},
number = {2},
pages = {830-846},
year = {2013},
doi = {10.1021/ie301386h},

URL = {https://doi.org/10.1021/ie301386h},
eprint = {https://doi.org/10.1021/ie301386h}
}

@article{araujo08,
title = "Control structure design for the ammonia synthesis process",
journal = "Computers \& Chemical Engineering",
volume = "32",
number = "12",
pages = "2920 - 2932",
year = "2008",
issn = "0098-1354",
doi = "https://doi.org/10.1016/j.compchemeng.2008.03.001",
url = "http://www.sciencedirect.com/science/article/pii/S0098135408000392",
author = "Antonio Araújo and Sigurd Skogestad",
keywords = "Ammonia plant, Control structure, Top-down analysis, Bottom-up design, Bottleneck, Throughput, Primary controlled variable, Secondary controlled variables, Manipulable variables",
abstract = "This paper discusses the application of the plantwide control design procedure of Skogestad [Skogestad, S. (2004a). Control structure design for complete chemical plants. Computers and Chemical Engineering, 28, 219–234] to the ammonia synthesis process. Three modes of operation are considered: (I) given feed rate, (IIa) maximum throughput, and (IIb) “optimized” throughput. Two control structures, one for Mode I and another for Mode IIb, are proposed. In Mode I, it is proposed to keep constant purge rate and compressor powers. There is no bottleneck in the process, and thus there is no Mode IIa of operation. In Mode IIb, the compressors are at their maximum capacity and it is proposed to adjust the feed rate such that the inert concentration is constant. The final control structures result in good dynamic performance."
}

@incollection{minasidis15,
title = "Simple Rules for Economic Plantwide Control",
editor = "Krist V. Gernaey and Jakob K. Huusom and Rafiqul Gani",
series = "Computer Aided Chemical Engineering",
publisher = "Elsevier",
volume = "37",
pages = "101 - 108",
year = "2015",
booktitle = "12th International Symposium on Process Systems Engineering and 25th European Symposium on Computer Aided Process Engineering",
issn = "1570-7946",
doi = "https://doi.org/10.1016/B978-0-444-63578-5.50013-X",
url = "http://www.sciencedirect.com/science/article/pii/B978044463578550013X",
author = "Vladimiros Minasidis and Sigurd Skogestad and Nitin Kaistha",
keywords = "Self-Optimizing Control, Economic Plantwide Control, Advanced Process Control",
abstract = "In this work, we consider the systematic economic plantwide control design procedure proposed by Skogestad (2004) and from this we derive practical rules that can be used to devise close-to-optimal control structures based on engineering insight. We attempt to present these rules in an easy-to-understand fashion and exemplify them on a simple but quite famous reactor-separator-recycle plant case study. We successfully demonstrate that while Skogestad’s procedure requires an optimization of the plant model under various disturbances in order to be fully utilized, using its practical rules, combined with a good engineering insight of the process, can facilitate the design of a close-to-optimal control structure by suggesting what should be controlled and how."
}

@article{jacobsen11,
author = {Jacobsen, Magnus G. and Skogestad, Sigurd},
title = {Active Constraint Regions for Optimal Operation of Chemical Processes},
journal = {Industrial \& Engineering Chemistry Research},
volume = {50},
number = {19},
pages = {11226-11236},
year = {2011},
doi = {10.1021/ie2012196},

URL = {https://doi.org/10.1021/ie2012196},
eprint = {https://doi.org/10.1021/ie2012196}
}


@article{quirante16,
title = "Large scale optimization of a sour water stripping plant using surrogate models",
journal = "Computers \& Chemical Engineering",
volume = "92",
pages = "143 - 162",
year = "2016",
issn = "0098-1354",
doi = "https://doi.org/10.1016/j.compchemeng.2016.04.039",
url = "http://www.sciencedirect.com/science/article/pii/S0098135416301351",
author = "Natalia Quirante and José A. Caballero",
keywords = "Process simulation, Process optimization, Kriging interpolation, Heat exchanger network, Life cycle assessment",
abstract = "In this work, we propose a new methodology for the large scale optimization and process integration of complex chemical processes that have been simulated using modular chemical process simulators. Units with significant numerical noise or large CPU times are substituted by surrogate models based on Kriging interpolation. Using a degree of freedom analysis, some of those units can be aggregated into a single unit to reduce the complexity of the resulting model. As a result, we solve a hybrid simulation-optimization model formed by units in the original flowsheet, Kriging models, and explicit equations. We present a case study of the optimization of a sour water stripping plant in which we simultaneously consider economics, heat integration and environmental impact using the ReCiPe indicator, which incorporates the recent advances made in Life Cycle Assessment (LCA). The optimization strategy guarantees the convergence to a local optimum inside the tolerance of the numerical noise."
}


@manual{aspentech,
title = "Aspen Plus User Guide",
author = {AspenTech},
year = {2017},
language = {English},
version = {Version 10},
organization = {AspenTech Inc.},
}


@article{jagtap13,
author = {Jagtap, Rahul and Pathak, Ashok S and Kaistha, Nitin},
title = {Economic plantwide control of the ethyl benzene process},
journal = {AIChE Journal},
volume = {59},
number = {6},
pages = {1996-2014},
keywords = {plantwide control, economic operation, bottleneck, capacity constraint control, override control},
doi = {10.1002/aic.13964},
url = {https://aiche.onlinelibrary.wiley.com/doi/abs/10.1002/aic.13964},
eprint = {https://aiche.onlinelibrary.wiley.com/doi/pdf/10.1002/aic.13964},
abstract = {Systematic plantwide control system design for economically optimal operation of the ethyl benzene process over a large throughput range is studied. As throughput is increased, constraints progressively become active with the highest number of active constraints at maximum throughput. An economic plantwide control system (CS1) is designed for operation at this most constrained operating point using a novel “top-down” pairing approach with higher prioritization to the economic objectives over regulatory objectives. This structure is adapted for near optimal low throughput operation with constraints that go inactive taking up additional economic variable control. For comparison, a conventional plantwide control structure (CS2) with the throughput manipulator at a fresh feed and “bottom-up” pairing for the control objectives is also synthesized. Four overrides are needed in CS2 to handle the hard equipment capacity constraints at maximum throughput. Rigorous dynamic simulations show that CS1 is dynamically and economically significantly superior to CS2. © 2012 American Institute of Chemical Engineers AIChE J, 59: 1996–2014, 2013},
year = {2013}
}

@article{jagtap12,
author = {Jagtap, Rahul and Kaistha, Nitin},
title = {Economic Plantwide Control of a C4 Isomerization Process},
journal = {Industrial & Engineering Chemistry Research},
volume = {51},
number = {36},
pages = {11731-11743},
year = {2012},
doi = {10.1021/ie3001293},
URL = {https://doi.org/10.1021/ie3001293},
eprint = { 
        https://doi.org/10.1021/ie3001293}

}

@article{sidinei2017,
author = {Silva, S. and Villar, S. and Costa, A. and Teixeira, H. and Araújo, A.},
year = {2017},
month = {07},
pages = {851-871},
title = {Development and application of an automatic tool for the selection of control variables based on the self-optimizing control methodology},
volume = {34},
journal = {Brazilian Journal of Chemical Engineering},
doi = {10.1590/0104-6632.20170343s20150445}
}

@Article{Cozad2014,
  author    = {Alison Cozad and Nikolaos V. Sahinidis and David C. Miller},
  title     = {Learning surrogate models for simulation-based optimization},
  journal   = {{AIChE} Journal},
  year      = {2014},
  volume    = {60},
  number    = {6},
  pages     = {2211--2227},
  month     = {mar},
  doi       = {10.1002/aic.14418},
  publisher = {Wiley},
}

@Article{Boukouvala2016,
  author    = {Fani Boukouvala and Christodoulos A. Floudas},
  title     = {{ARGONAUT}: {AlgoRithms} for Global Optimization of {coNstrAined} grey-box {compUTational} problems},
  journal   = {Optimization Letters},
  year      = {2016},
  volume    = {11},
  number    = {5},
  pages     = {895--913},
  month     = {apr},
  doi       = {10.1007/s11590-016-1028-2},
  publisher = {Springer Science and Business Media {LLC}},
}

@Article{gorissen10,
  author  = {Dirk Gorissen and Ivo Couckuyt and Piet Demeester and Tom Dhaene and Karel Crombecq},
  title   = {A Surrogate Modeling and Adaptive Sampling Toolbox for Computer Based Design},
  journal = {Journal of Machine Learning Research},
  year    = {2010},
  volume  = {11},
  number  = {68},
  pages   = {2051-2055},
  url     = {http://jmlr.org/papers/v11/gorissen10a.html},
}

@Comment{jabref-meta: databaseType:bibtex;}
